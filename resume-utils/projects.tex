\def\projectHoloFr{
\makeproject{
title=Projet d'école,
desc={
\textit{Holographie numérique et deep learning pour la localisation 3D de bactéries}\par
J’ai étudié des méthodes de deep learning pour le suivi 3D de bactéries à partir de motifs de diffraction holographiques. J’ai implémenté et entraîné un modèle basé sur la physique à l'aide de données synthétiques simulées, puis évalué ses performances sur des données expérimentales du LEMTA.
% \begin{itemize}
%     \item Étude des méthodes de deep learning pour le suivi 3D à partir de motifs de diffraction holographiques.
%     \item Implémentation et entraînement d’un modèle de localisation sur des données synthétiques simulées.
%     \item Évaluation des performances sur des données expérimentales du LEMTA.
% \end{itemize}
}
}}

\def\projectMlgrindFr{
\makeproject{
title=Projet personnel,
desc={
    \textit{Machine Learning grind}, \url{https://github.com/bmalick/machine-learning-grind}\par
J'étudie des articles de recherche en machine learning et deep learning. J'implémente \textit{from scratch} plusieurs algorithmes statistiques et de deep learning. J'applique ces modèles à des jeux de données pour en évaluer l’efficacité.
% \begin{itemize}
%     \item Étude d’articles de recherche
    % \item Lecture de livres de Deep learning pour une compréhension détaillée
    % \item Implémentation des algorithmes \textit{from scratch} de modèles de \textit{deep learning} et de statistiques
    % \item Applications des modèles étudiés
% \end{itemize}
}
}}

\def\projectHoloEn{
\makeproject{
title=School project,
desc={
\textit{Digital holography and deep learning for 3D localization of bacteria}\par
I studied deep learning methods for 3D tracking of bacteria using holographic diffraction patterns. I implemented and trained a physics-based localization model using simulated synthetic data, then evaluated its performance on experimental data provided by LEMTA.
% \begin{itemize}
%     \item Study of deep learning methods for 3D tracking from holographic diffraction patterns.
%     \item Implementation and training of a localization model on simulated synthetic data.
%     \item Performance evaluation on experimental data from LEMTA.
% \end{itemize}
}
}}

\def\projectMlgrindEn{
\makeproject{
title=Personal project,
desc={
    \textit{Machine Learning grind}, \url{https://github.com/bmalick/machine-learning-grind}\par
I study research papers in machine learning and deep learning. I implement various statistical and deep learning algorithms from scratch. I apply these models to datasets to evaluate their performance.
% \begin{itemize}
%     \item Study of research articles in machine learning and deep learning.
%     \item Implementation of statistical and deep learning models from scratch.
%     \item Application of the studied models to datasets for performance evaluation.
% \end{itemize}
}
}}


\def\projectSemEightFr{
\makeproject{
title=Projet d'école,
desc={
\textit{Débruitage d'images simulées de stents}\par
J’ai travaillé sur la restauration d’images à rayons X présentant un fort niveau de bruit. J'ai simulé des images bruitées contenant des stents. J’ai implémenté une architecture U-Net pour effectuer le débruitage, comparé avec des méthodes de réduction de dimension comme le \textit{PCA} et {BM3D}. J'ai aussi appliqué des techniques d’augmentation de données.
% \begin{itemize}
%     \item Restauration de séquences d’images à rayons X à fort niveau de bruit
%     \item Restauration d'images simulées de stents
%     \item Implémentation d'une architecture U-Net
%     \item Comparaison de différentes méthodes de débruitage
%     \item Data Augmentation avec albumentations
% \end{itemize}
}
}}

\def\projectSemEightEn{
\makeproject{
title=School project,
desc={
\textit{Denoising of simulated stent images}\par
I worked on the restoration of X-ray images with high noise levels. I simulated noisy images that contain stents. I implemented a U-Net architecture for the denoising task, compared different denoising methods like PCA and BM3D. I also applied data augmentation techniques.
}
}}

