\def\projectHoloFr{
\makeproject{
title={Projet d'école : Holographie numérique et deep learning pour la localisation 3D de bactéries},
desc={J'ai étudié des méthodes de deep learning pour le suivi 3D de bactéries à partir de motifs de diffraction holographiques. J'ai implémenté et entraîné un modèle basé sur la physique à l'aide de données synthétiques simulées, puis évalué ses performances sur des données expérimentales du LEMTA.}
}}

\def\projectMlgrindFr{
\makeproject{
title={Projet personnel : Machine Learning grind, \url{https://github.com/bmalick/machine-learning-grind}},
desc={J'étudie des articles de recherche en machine learning et deep learning. J'implémente \textit{from scratch} plusieurs algorithmes statistiques et de deep learning. J'applique ces modèles à des jeux de données afin d'en évaluer l'efficacité.}
}}

\def\projectHoloEn{
\makeproject{
title={School project : Digital holography and deep learning for 3D localization of bacteria},
desc={I studied deep learning methods for 3D tracking of bacteria using holographic diffraction patterns. I implemented and trained a physics-based localization model using simulated synthetic data, then evaluated its performance on experimental data from LEMTA.}
}}

\def\projectMlgrindEn{
\makeproject{
title={Personal project : Machine Learning grind, \url{https://github.com/bmalick/machine-learning-grind}},
desc={I study research papers in machine learning and deep learning. I implement various statistical and deep learning algorithms from scratch. I apply these models to datasets to evaluate their performance.}
}}

\def\projectSemEightFr{
\makeproject{
title={Projet d'école : Débruitage d'images simulées de stents},
desc={J'ai travaillé sur la restauration d'images à rayons X présentant un fort niveau de bruit. J'ai simulé des images bruitées contenant des stents. J'ai implémenté une architecture U-Net pour effectuer le débruitage et l'ai comparée à des méthodes de réduction de dimension comme le \textit{PCA} et \textit{BM3D}. J'ai également appliqué des techniques d'augmentation des données.}
}}

\def\projectSemEightEn{
\makeproject{
title={School project : Denoising of simulated stent images},
desc={I worked on the restoration of X-ray images with high noise levels. I simulated noisy images containing stents. I implemented a U-Net architecture for the denoising task and compared it to different methods such as PCA and BM3D. I also applied data augmentation techniques.}
}}


\def\projectIartFr{
\makeproject{
title={Projet d'école : iArt — Reconnaissance visuelle de tableaux},
desc={Mini bootcamp de programmation visant à créer une application web de reconnaissance de tableaux à partir d'images.}
}}


\def\projectIartEn{
\makeproject{
title={School project : iArt — Painting recognition},
desc={Mini programming bootcamp involving the creation of a web application for painting recognition from images.}
}}


\def\projectSentimentFr{
\makeproject{
title={Projet d'école : Analyse de sentiments},
desc={Scraping de tweets via l'API Twitter autour du procès Johnny Depp / Amber Heard. Classification des tweets pour détecter le soutien à l'un ou l'autre des protagonistes. Utilisation de réseaux de neurones récurrents, ainsi que de techniques de tokenisation et lemmatisation.}
}}


\def\projectSentimentEn{
\makeproject{
title={School project : Sentiment analysis},
desc={Scraping tweets using the Twitter API about the Johnny Depp / Amber Heard trial. Classification of tweets to detect support for either side. Use of recurrent neural networks, tokenization, and lemmatization techniques.}
}}


\def\projectFootFr{
\makeproject{
title={Projet personnel : Automatisation de la création de pages de matchs de football},
desc={Automatisation de la génération de pages Notion via l'API Notion. Web scraping des données de matchs et programmation orientée objet pour structurer les données.}
}}


\def\projectFootEn{
\makeproject{
title={Personal project : Automated generation of football match pages},
desc={Automation of Notion page creation via the Notion API. Web scraping of match data and use of object-oriented programming to structure the information.}
}}


\def\projectTweetoscopeFr{
\makeproject{
title={Projet d'école : Tweetoscope -- Application Java de traitement de tweets},
desc={Utilisation de Kafka et conteneurisation des services. Déploiement sur un cluster Kubernetes et analyse des risques et scénarios de défaillance.}
}}


\def\projectTweetoscopeEn{
\makeproject{
title={School project : Tweetoscope — Java application for tweet processing},
desc={Use of Kafka and containerization of services. Deployment on a Kubernetes cluster and analysis of risks and failure scenarios.}
}}


