% [Michelin]
\def\xpMichelinFr{
\makexp{
post=Stage de recherche,
company=Michelin,
location=Clermont-Ferrand,
duration=Mai 2025 - Nov 2025,
desc={
\textit{Réduction d’artefacts métalliques sur les images tomographiques de pneumatiques par Deep Learning}\par
Après une formation à la tomographie par rayons X, j'ai réalisé un état de l’art sur la réduction des artefacts métalliques par \textit{deep learning}. J'ai constitué une première base de données synthétique d’images tomographiques de pneus. J'ai réalisé un benchmark de modèles les plus prometteurs proposant une approche supervisée et une approche non supervisée.
% \begin{itemize}
%     \item Formation à la tomographie par rayons X
%     \item Etat de l’art sur la réduction des artefacts métalliques
%     \item Création de base de données tomographiques de pneus
%     \item Modèles de diffusion
% \end{itemize}
}}}

\def\xpMichelinEn{
\makexp{
post=Research Internship,
company=Michelin,
location=Clermont-Ferrand,
duration=May 2025 – Nov 2025,
desc={
\textit{Reduction of metal artifacts in tire tomographic images using Deep Learning}\par
After a quick learning on X-ray tomography, I conducted a literature review on metal artifact reduction using deep learning techniques. I built a synthetic tomographic image database of tires. I made a benchmark of most promising models based on supervised and unsupervised learning.
% \begin{itemize}
%     \item Training in X-ray tomography
%     \item Literature review on metal artifact reduction
%     \item Creation of a tomographic image database of tires
%     \item Use of diffusion models
% \end{itemize}
}
}}
% [Michelin]

% [EDF]
\def\xpEdfFr{
\makexp{
post=Stage,
company=EDF,
location=Grenoble,
duration=Mars 2024 - Août 2024,
desc={
\textit{Classifications des défaillances de turbines}\par
J’ai annoté des données textuelles liées aux défaillances des turbines, conçu un guide méthodologique pour l’annotation, et fine-tuné le modèle de langage \textit{CamemBERT} pour cette tâche spécifique.
% \begin{itemize}
%     \item Annotation des données textuelles relatives aux défaillances des turbines
%     \item Élaboration d’un guide méthodologique d’annotation
%     \item Fine-tuning du modèle CamemBERT
% \end{itemize}
}}}

\def\xpEdfEn{
\makexp{
post=Internship,
company=EDF,
location=Grenoble,
duration=March 2024 – August 2024,
desc={
\textit{Turbine failure classification}\par
I annotated textual data related to turbine failures, designed a methodological annotation guide, and fine-tuned the CamemBERT language model for this specific task.
% \begin{itemize}
%     \item Annotation of textual data related to turbine failures
%     \item Design of a methodological annotation guide
%     \item Fine-tuning of the CamemBERT model
% \end{itemize}
}
}}
% [EDF]

% [Innov-plus]
\def\xpInnovplusFr{
\makexp{
post=Stage de recherche,
company=Innov+,
location=Gif-sur-Yvette,
duration=Juil 2023 - Dec 2023,
desc={
\textit{Prédiction d’émotions à partir de vidéos}\par
J’ai mené une étude de l’état de l’art, implémenté des architectures de convolutions 3D inspirées d’articles scientifiques, et développé des techniques de prétraitement et d’augmentation de données vidéo.
% \begin{itemize}
%     \item  Etude de l’état de l’art sur la prédiction des émotions
%     \item  Implémentation d’architectures de convolutions 3D issues d’articles de recherche
%     \item  Prétraitement et des données vidéo techniques d’augmentation \end{itemize}
}}}

\def\xpInnovplusEn{
\makexp{
post=Research Internship,
company=Innov+,
location=Gif-sur-Yvette,
duration=July 2023 – Dec 2023,
desc={
\textit{Emotion prediction from video data}\par
I conducted a literature review on emotion prediction, implemented 3D convolutional architectures based on research papers, and developed preprocessing and data augmentation techniques for video data.
% \begin{itemize}
%     \item Literature review on emotion prediction
%     \item Implementation of 3D convolutional architectures from research papers
%     \item Preprocessing and data augmentation of video data
% \end{itemize}
}
}}
% [Innov-plus]




\def\nameEn{
\makexp{
post=
company=
location=
duration=
desc={
}
}}


\def\nameFr{
\makexp{
post=
company=
location=
duration=
desc={

}}}
