\newcommand{\makeTweetoscope}[2]{% #1: resume type, #2: language
    \ifthenelse{\equal{#2}{fr}}{
        \tweetoscopeFr{#1}}{\tweetoscopeEn{#1}}
}

\newcommand{\tweetoscopeFr}[1]{% #1: Resume Type
    \ifthenelse{\equal{#1}{1}}{
        \cvevent{}{Projet d'école: Tweetoscope - application Java de traitement de tweets}{Oct-Nov 2024}{}
    }{
        \textbf{Tweetoscope - application Java de traitement de tweets} \hfill {\textbf{Projet d'école}, Oct-Nov 2024}
    }
    \begin{itemize}
        \item Utilisation de Kafka et conteneurisation des services 
        % \item Création d'un projet sur GitLab avec intégration CI/CD
        \item Déploiement sur un cluster Kubernetes
        \item Analyse des risques et gestion des pannes
        % \item Écriture de tests JUnit et intégration à la pipeline CI
    \end{itemize}
}

\newcommand{\tweetoscopeEn}[1]{% #1: Resume Type
    \ifthenelse{\equal{#1}{1}}{
        \cvevent{}{School project: Tweetoscope - Java Application for Processing Tweets}{Oct-Nov 2024}{}
    }{
        \textbf{Tweetoscope - Java Application for Processing Tweets} \hfill {\textbf{School Project}, Oct-Nov 2024}
    }
    \begin{itemize}
        \item Utilization of Kafka and containerization of services
        % \item Creation of a GitLab project with CI/CD integration
        \item Deployment on a Kubernetes cluster
        \item Risk analysis and failure management
        % \item Writing JUnit tests and integration into the CI pipeline
    \end{itemize}
}

