
\begin{flushright}
Baye Malick Gning\\
baye.gning@student-cs.fr\\
+33 7 61 13 27 49\\
Clermont-Ferrand, France
\end{flushright}

\vspace{1cm}

\noindent
\textbf{À l’attention du jury du Master MVA}\\
ENS Paris-Saclay

\vspace{0.5cm}

Actuellement en stage de fin d'étude pour mon diplôme de CentraleSupélec et du Master 2 de Mathématiques Appliquées de l’Université de Lorraine, je souhaite vous soumettre ma candidature au Master 2 « Mathématiques, Vision, Apprentissage » (MVA) de l’ENS Paris-Saclay. Je suis particulièrement intéressé par le \textit{parcours Santé}, qui représente pour moi une opportunité unique de poursuivre ma spécialisation en intelligence artificielle appliquée au biomédical et dans le domaine de la robotique (notamment avec les prothèses assiéte par IA), et de concrétiser un projet professionnel centré sur la recherche et l’entrepreneuriat en IA médicale.

Ma formation d’ingénieur m’a offert une base rigoureuse en science des données, traitement du signal, informatique et apprentissage automatique, que j’ai enrichie par un Master en mathématiques appliquées. J’y ai approfondi les probabilités avancées, l’optimisation, la statistique bayésienne et la théorie de l’estimation — des piliers essentiels pour maîtriser les fondements théoriques de l’apprentissage statistique et des modèles modernes d’IA. Cette double compétence ingénierie/maths me permet aujourd’hui d’aborder des problèmes complexes à la fois avec rigueur mathématique et intuition algorithmique.

Mon intérêt pour l’IA appliquée à la médecine s’est construit au fil de mes projets. Lors de mon projet de fin d’études, j’ai travaillé au laboratoire LEMTA sur un système de localisation 3D de bactéries à partir d’images holographiques, combinant physique de la diffraction et \textit{deep learning}. Ce projet n'a fait que confirmer mon intêret à l'application de mes compétences en IA en médecine.


J’ai également mené deux stages significatifs en IA. Chez Innov+, j’ai travaillé sur de la prédiction d’émotions à partir de vidéos pour des applications en sécurité routière, en m’appuyant sur des architectures 3D convolutives et une analyse fine des mouvements. Ce stage m’a sensibilisé à l’impact sociétal de l’IA et son potentiel impact dans la sauvegarde des conducteurs et de leurs passager. Chez EDF, j’ai réalisé un projet de classification de défauts de turbines via le \textit{fine-tuning} de CamemBERT sur des rapports techniques annotés, en élaborant en amont un guide méthodologique d’annotation pour les experts métier. Ce stage était pour moi l'occasion de découvrir les modèles de langages qui sont en vogue depuis peu de temps.

Aujourd’hui, je poursuis un stage de recherche en imagerie tomographique industrielle, avec un fort intérêt pour la reconstruction 3D et la correction d’artefacts métalliques — techniques directement transposables à l’imagerie médicale (IRM, CT-scan, radiologie). Bien que je sois dans le domaine industriel pour mon stage de fin d'étdues, tous ces travaux nourrissent un projet professionnel clair : contribuer à la recherche en IA pour la médecine, et à terme créer une startup en santé numérique, spécialisée dans l’IA appliquée au diagnostic par l’image et à la robotique.

Au-delà de mon stage de recherche, la lecture d'articles de recherche est une de mes préoccupations. Je travaille depuis des mois sur un projet que j'ai nommé \textit{machine learning grind}. Il s'agit d'une démarche structurée de formation autodidacte où je synthétise mes lectures en statistiques et particulièrement en \textit{deep learning} et implémente les architectures issues d'articles de recherche, particulièrement en vision par ordinateur.

Intégrer le MVA me permettrait de consolider mes compétences théoriques, notamment en apprentissage statistique, vision par ordinateur et traitement de données biomédicales, tout en bénéficiant d’un encadrement académique de haut niveau. Le parcours Santé, à travers des cours comme \textit{Deep Learning for Biomedical Data}, \textit{Medical Imaging} ou \textit{Geometry and Statistics in Imaging}, représente exactement la formation avancée que je recherche pour franchir un cap dans ma préparation à la réalisation d'une thèse. La suite sera la fondation d'une startup dans ce domaine. Ce projet est également le fruit de longs échanges avec une camarade que je connais depuis le lycée. Celui-ci travaille dans la robotique et nous pensons que nous pouvons avec nos connaissances se lancer à l'avenir dans des projets sur l'IA, la robotique et travailler sur des proojets concernant la santé.

Je vous remercie pour l’attention portée à ma candidature, et reste à votre disposition pour tout complément d’information.

\vspace{0.5cm}

\noindent
Veuillez recevoir, Madame, Monsieur, l’expression de mes salutations respectueuses.

\vspace{0.8cm}
