
\begin{flushright}
Baye Malick Gning\\
baye.gning@student-cs.fr\\
+33 7 61 13 27 49\\
Metz, France
\end{flushright}

\vspace{1cm}

\noindent
\textbf{Objet} : Candidature au poste de Data Scientist / Ingénieur IA – Imagerie médicale

\vspace{0.5cm}

Madame, Monsieur,

Ingénieur diplômé de CentraleSupélec et titulaire d’un Master en Mathématiques Appliquées, je me spécialise depuis plusieurs années dans l’intelligence artificielle appliquée à la santé, avec une expertise particulière en traitement d’images et en apprentissage automatique. C’est dans ce cadre que je souhaite rejoindre votre équipe.

Mon profil allie une base mathématique solide (probabilités, optimisation, statistique bayésienne) à une forte capacité à développer des solutions concrètes en deep learning. Lors de mon stage de fin d’études au laboratoire LEMTA, j’ai conçu un système de localisation 3D de bactéries à partir d’images holographiques. Ce projet, au croisement de la physique et de l’IA, m’a permis d’explorer des techniques avancées de reconstruction spatiale et d’apprentissage supervisé.

En parallèle, j’ai mené deux projets dans des environnements industriels exigeants. Chez Innov+, j’ai développé un modèle 3D convolutif pour prédire les émotions dans des vidéos, dans le cadre d’un projet de sécurité routière. Chez EDF, j’ai conçu une chaîne de traitement NLP pour classer des rapports techniques via le fine-tuning de CamemBERT, tout en contribuant à la méthodologie d’annotation des données.

Mon autonomie est également visible dans un projet personnel de longue haleine : \textbf{Machine Learning Grind}, une plateforme que j’ai créée pour structurer et partager mon apprentissage approfondi de l’IA, depuis les bases statistiques jusqu’aux architectures de vision et de traitement du langage.

Je suis convaincu que mon profil polyvalent, alliant rigueur scientifique, curiosité, et expérience appliquée, me permettrait de contribuer efficacement à vos projets en intelligence artificielle pour la santé. Je serais heureux d’échanger avec vous afin d'envisager une collaboration.

Je vous remercie pour l’attention portée à ma candidature,\\
Bien cordialement,
