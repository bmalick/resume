% =======================
% Medical - general
% =======================
\def\introductionParaphMedicalFr{
Je serai diplômé de CentraleSupélec cette année, tout en étant titulaire d'un double diplôme avec le Master de recherche en Mathématiques Fondamentales Appliquées de l’Université de Lorraine. Ma spécialisation porte sur l’intelligence artificielle, et plus précisément sur ses applications en vision par ordinateur (\textit{computer vision}) dans le domaine médical. Je souhaite ainsi intégrer votre équipe pour le poste de \postName, afin de contribuer à vos travaux à l’intersection de l’IA et de la santé.
}

\def\interestPostMedicalFr{
Votre projet de recherche m’intéresse particulièrement en raison de son application au domaine de la santé. Je suis convaincu que l’intelligence artificielle jouera un rôle majeur dans les avancées médicales des prochaines années, et je souhaite contribuer à cette évolution en y apportant mes compétences et ma passion.

Je suis également très motivé par la perspective d’évoluer dans un environnement pluridisciplinaire, où la rigueur scientifique se conjugue à un fort impact sociétal. Participer à un projet aussi ambitieux représente pour moi une opportunité stimulante de mettre mes connaissances en IA au service de problématiques de santé concrètes.
}

\def\whyMeMedicalFr{
Ma double formation en ingénierie à CentraleSupélec et en mathématiques fondamentales appliquées à l’Université de Lorraine m’a permis d’acquérir une solide compréhension théorique et pratique de l’intelligence artificielle. J’ai notamment développé des projets en vision par ordinateur (\textit{computer vision}) appliquée au domaine médical, ce qui m’a sensibilisé aux enjeux spécifiques de ce secteur.

Je suis autonome, curieux, et j’aime approfondir mes connaissances par la pratique ainsi que par la lecture d’ouvrages et d’articles sur l’IA appliquée à la santé. Je m'intéresse particulièrement aux revues comme \textit{Nature Biomedical Engineering}, \textit{npj Digital Medicine} ou encore \textit{The Lancet Digital Health}, qui m’ont permis de mieux comprendre l’impact des techniques d’IA dans le diagnostic médical et la médecine personnalisée. Je serais ravi de développer ces points lors d’un entretien. Je suis également à l’aise avec les outils de développement modernes (Python, Git, Docker, etc.), ce qui me permet de m’intégrer rapidement à un environnement technique et collaboratif.
}

\def\skillsMedicalFr{
Au fil de mon parcours, j’ai développé de solides compétences en mathématiques appliquées, notamment en probabilités et statistiques avancées, en optimisation, ainsi qu’en apprentissage automatique (\textit{machine learning}). Ces compétences ont été consolidées à travers mon master de recherche en mathématiques fondamentales appliquées et ma spécialisation en science des données et de l’information à CentraleSupélec.

Je maîtrise les principaux frameworks de \textit{machine learning} et \textit{deep learning} (Scikit-Learn, \textit{PyTorch}, \textit{TensorFlow}), ainsi que des bibliothèques spécifiques au traitement d’images médicales (\textit{OpenCV}, \textit{Albumentations}). Ces outils m’ont permis de contribuer à plusieurs projets liés à la santé : localisation 3D de bactéries par holographie numérique (LEMTA), débruitage d’images de stents simulées, et réduction d’artefacts métalliques sur images tomographiques lors de mon stage chez Michelin.

Mon projet personnel \textit{Machine Learning Grind}, que je maintiens sur GitHub, reflète mon investissement profond dans la compréhension et l’expérimentation des modèles d’IA.
}

\def\conclusionMedicalFr{
Je serais ravi de pouvoir échanger avec vous sur votre projet de recherche et la manière dont je pourrais y contribuer. Je reste bien entendu à votre disposition pour un entretien ou toute information complémentaire.

Je vous remercie pour l’attention portée à ma candidature.
}

% =======================
% Medical - general En
% =======================
\def\introductionParaphMedicalEn{
I will graduate this year from CentraleSupélec, while also completing a double degree with the Research Master's in Applied Fundamental Mathematics at the University of Lorraine. My specialization focuses on artificial intelligence, and more specifically its applications in computer vision for the medical field. I would like to join your team for the position of \postName, and contribute to your work at the intersection of AI and healthcare.
}

\def\interestPostMedicalEn{
I am particularly interested in your research project due to its application to the healthcare domain. I am convinced that artificial intelligence will play a major role in medical advances in the coming years, and I would like to contribute to this evolution by applying my skills and passion.

I am also very motivated by the opportunity to work in a multidisciplinary environment where scientific rigor goes hand-in-hand with societal impact. Being part of such an ambitious project represents an exciting opportunity to apply my AI knowledge to real healthcare challenges.
}

\def\whyMeMedicalEn{
My dual education in engineering at CentraleSupélec and applied fundamental mathematics at the University of Lorraine has given me a strong theoretical and practical understanding of artificial intelligence. I have developed several computer vision projects applied to the medical domain, which have made me more aware of the specific challenges in this field.

I am independent, curious, and I enjoy deepening my knowledge through both practice and reading scientific literature on AI in healthcare. I regularly follow journals such as Nature Biomedical Engineering, npj Digital Medicine, and The Lancet Digital Health, which have helped me better understand the impact of AI techniques on diagnostics and personalized medicine. I would be glad to elaborate on these points during an interview. I am also proficient with modern development tools (Python, Git, Docker, etc.), enabling me to integrate effectively into technical and collaborative environments.
}

\def\skillsMedicalEn{
Throughout my studies, I have developed strong skills in applied mathematics, including advanced probability and statistics, optimization, and machine learning. These competencies were reinforced through my research master’s in applied mathematics and my specialization in data science and information systems at CentraleSupélec.

I am proficient in major machine learning and deep learning frameworks (Scikit-Learn, PyTorch, TensorFlow), as well as libraries specific to medical image processing (OpenCV, Albumentations). These tools have enabled me to contribute to several healthcare-related projects: 3D localization of bacteria via digital holography (LEMTA), denoising of simulated stent images, and the reduction of metal artifacts in tomographic images during my research internship at Michelin.

My personal project, \textit{Machine Learning Grind}, which I maintain on GitHub, reflects my deep commitment to understanding and experimenting with AI models.
}

\def\conclusionMedicalEn{
I would be delighted to discuss your research project and how I might contribute to it. I remain available for any further information or an interview.

Thank you for considering my application.
}


% =======================
% Thesis - general
% =======================
\def\introductionParaphThesisFr{
Je serai diplômé de CentraleSupélec cette année, avec un double diplôme en Mathématiques Fondamentales Appliquées de l’Université de Lorraine. Je souhaite poursuivre en thèse dans le domaine de l’intelligence artificielle (\textit{artificial intelligence}) et du traitement des données, afin de développer mes compétences en recherche fondamentale et appliquée.
}

\def\interestPostThesisFr{
Le sujet de thèse que vous proposez m’intéresse particulièrement, car il correspond à mes aspirations scientifiques et à mes compétences en IA et data science. Je suis motivé par la perspective de mener un travail approfondi, original et rigoureux, contribuant à l’avancée des connaissances dans ce domaine.

Intégrer votre laboratoire serait une opportunité unique pour apprendre et contribuer à des projets de recherche ambitieux.
}

\def\whyMeThesisFr{
Ma double formation d’ingénieur et de chercheur m’a donné une solide base en mathématiques, programmation et intelligence artificielle. Je suis autonome, curieux, et je m’investis pleinement dans mes projets personnels et académiques, notamment en vision par ordinateur.

Je souhaite approfondir mes connaissances et développer mes compétences en recherche, en bénéficiant de votre encadrement et des ressources de votre laboratoire.
}

\def\skillsThesisFr{
J’ai acquis des compétences avancées en apprentissage automatique, optimisation et traitement des données, renforcées par mes études et projets. Je maîtrise \textit{Scikit-Learn}, \textit{PyTorch}, \textit{TensorFlow} et les outils modernes de développement, ce qui me permettra de mener efficacement mes travaux de recherche.

Mon projet personnel \textit{Machine Learning Grind}, que je maintiens sur GitHub, reflète mon investissement profond dans la compréhension et l’expérimentation des modèles d’IA.
}

\def\conclusionThesisFr{
Je serais honoré de pouvoir discuter plus en détail de cette thèse et de ma motivation. Je reste disponible pour toute information complémentaire ou entretien.

Je vous remercie pour l’attention portée à ma candidature.
}

% =======================
% Thesis - medical En
% =======================
\def\introductionParaphThesisMedicalFr{
Je serai diplômé de CentraleSupélec cette année, tout en étant titulaire d'un double diplôme avec le Master de recherche en Mathématiques Fondamentales Appliquées de l’Université de Lorraine. Ma spécialisation porte sur l’intelligence artificielle, et plus précisément sur ses applications en vision par ordinateur (\textit{computer vision}) dans le domaine médical. Je souhaite ainsi intégrer votre équipe pour le poste de \postName, afin de contribuer à vos travaux de recherche à l’intersection de l’IA et de la santé.
}

\def\interestPostThesisMedicalFr{
Le sujet de thèse que vous proposez m’intéresse profondément, notamment pour son application aux défis médicaux et de santé publique. Convaincu que l’intelligence artificielle jouera un rôle clé dans les avancées médicales à venir, je souhaite apporter mes compétences et ma passion pour contribuer activement à cette transformation. Je suis motivé par l’opportunité de participer à des progrès scientifiques concrets en médecine, par le développement de méthodes d’IA innovantes.

Intégrer votre équipe de recherche serait pour moi une chance unique de progresser dans ce domaine passionnant.
}


\def\whyMeThesisMedicalFr{
Ma double formation en ingénierie à CentraleSupélec et en mathématiques fondamentales appliquées à l’Université de Lorraine m’a permis d’acquérir une solide compréhension théorique et pratique de l’intelligence artificielle. J’ai notamment développé des projets en vision par ordinateur (\textit{computer vision}) appliquée au domaine médical, ce qui m’a sensibilisé aux enjeux spécifiques de ce secteur.

Je suis autonome, curieux, et j’aime approfondir mes connaissances par la pratique ainsi que par la lecture d’ouvrages et d’articles sur l’IA appliquée à la santé. Je m'intéresse particulièrement aux revues comme \textit{Nature Biomedical Engineering}, \textit{npj Digital Medicine} ou encore \textit{The Lancet Digital Health}, qui m’ont permis de mieux comprendre l’impact des techniques d’IA dans le diagnostic médical et la médecine personnalisée. Je serais ravi de développer ces points lors d’un entretien. Je suis également à l’aise avec les outils de développement modernes (Python, Git, Docker, etc.), ce qui me permet de m’intégrer rapidement à un environnement technique et collaboratif.
}

\def\skillsThesisMedicalFr{
Au fil de mon parcours, j’ai développé de solides compétences en mathématiques appliquées, notamment en probabilités et statistiques avancées, en optimisation, ainsi qu’en apprentissage automatique (\textit{machine learning}). Ces compétences ont été consolidées à travers mon master de recherche en mathématiques fondamentales appliquées et ma spécialisation en science des données et de l’information à CentraleSupélec.

Je maîtrise les principaux frameworks de \textit{machine learning} et \textit{deep learning} (Scikit-Learn, \textit{PyTorch}, \textit{TensorFlow}), ainsi que des bibliothèques spécifiques au traitement d’images médicales (\textit{OpenCV}, \textit{Albumentations}). Ces outils m’ont permis de contribuer à plusieurs projets liés à la santé : localisation 3D de bactéries par holographie numérique (LEMTA), débruitage d’images de stents simulées, et réduction d’artefacts métalliques sur images tomographiques lors de mon stage chez Michelin.

Mon projet personnel \textit{Machine Learning Grind}, que je maintiens sur GitHub, reflète mon investissement profond dans la compréhension et l’expérimentation des modèles d’IA.
}

\def\conclusionThesisMedicalFr{
Je serais très heureux de pouvoir discuter avec vous de ce projet de thèse et de mes motivations. Je reste disponible pour un entretien ou toute information complémentaire.

Je vous remercie pour l’attention portée à ma candidature.
}



% =======================
% Thesis – Medical En
% =======================
\def\introductionParaphThesisMedicalEn{
I will graduate from CentraleSupélec this year, while also completing a double degree with the Research Master's in Applied Fundamental Mathematics from the University of Lorraine. My specialization focuses on artificial intelligence, and more specifically on its applications in computer vision for the medical field. I am therefore eager to join your team for the position of \postName, in order to contribute to your research at the intersection of AI and healthcare.
}

\def\interestPostThesisMedicalEn{
The PhD topic you are proposing is of great interest to me, particularly due to its application to medical and public health challenges. I am convinced that artificial intelligence will play a key role in future medical breakthroughs, and I wish to contribute to this transformation by bringing in my skills and passion. I am driven by the opportunity to participate in scientific progress with real-world impact in medicine, through the development of innovative AI methods.

Joining your research team would be a unique opportunity for me to grow in this exciting field.
}

\def\whyMeThesisMedicalEn{
My dual academic background in engineering at CentraleSupélec and in applied fundamental mathematics at the University of Lorraine has provided me with a strong theoretical and practical understanding of artificial intelligence. I have worked on several projects in computer vision applied to the medical field, which has made me more aware of the specific challenges in this domain.

I am autonomous, curious, and enjoy deepening my knowledge both through hands-on projects and by reading books and scientific articles on AI in healthcare. I am particularly interested in journals such as \textit{Nature Biomedical Engineering}, \textit{npj Digital Medicine}, and \textit{The Lancet Digital Health}, which have helped me better understand the impact of AI techniques in medical diagnostics and personalized medicine. I would be delighted to further discuss these points in an interview. I am also comfortable with modern development tools (Python, Git, Docker, etc.), allowing me to quickly integrate into technical and collaborative environments.
}

\def\skillsThesisMedicalEn{
Throughout my academic journey, I have developed strong skills in applied mathematics, particularly in advanced probability and statistics, optimization, and machine learning. These skills were strengthened through my research master’s in applied fundamental mathematics and my specialization in data and information science at CentraleSupélec.

I am proficient with major machine learning and deep learning frameworks (Scikit-Learn, \textit{PyTorch}, \textit{TensorFlow}), as well as libraries specific to medical image processing (\textit{OpenCV}, \textit{Albumentations}). These tools have enabled me to contribute to several healthcare-related projects: 3D localization of bacteria using digital holography (LEMTA), denoising of simulated stent images, and metal artifact reduction in tomographic imaging during my internship at Michelin.

My personal project, \textit{Machine Learning Grind}, which I maintain on GitHub, reflects my deep commitment to understanding and experimenting with AI models.
}

\def\conclusionThesisMedicalEn{
I would be delighted to discuss this PhD project and my motivations with you. I remain available for an interview or to provide any additional information.

Thank you for considering my application.
}
