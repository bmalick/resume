
\begin{flushright}
Baye Malick Gning\\
baye.gning@student-cs.fr\\
+33 7 61 13 27 49\\
Metz, France
\end{flushright}

\vspace{1cm}

\noindent
\textbf{Objet} : Candidature à une thèse en intelligence artificielle appliquée à l’imagerie médicale

\vspace{0.5cm}

Madame, Monsieur,

Diplômé de CentraleSupélec et du Master 2 de Mathématiques Appliquées de l’Université de Lorraine, je souhaite aujourd’hui m’engager dans une thèse en intelligence artificielle, avec un fort intérêt pour ses applications à la santé, notamment dans le traitement et l’analyse d’images médicales.

Mon double parcours m’a permis de construire des bases solides, à la fois théoriques et pratiques. Mon Master m’a formé à des domaines fondamentaux pour l’apprentissage automatique tels que l’optimisation, les probabilités avancées, l’inférence bayésienne et la théorie de l’estimation. En parallèle, ma formation d’ingénieur m’a exposé à des projets concrets, et m’a permis de développer une approche rigoureuse des systèmes de traitement de données et d’images.

Durant mon projet de fin d’études au laboratoire LEMTA, j’ai développé un modèle de localisation 3D de bactéries à partir d’images holographiques, combinant techniques physiques et apprentissage profond. Ce projet s’inscrit dans la continuité d’un travail personnel d’autoformation que je mène depuis plusieurs mois, documenté publiquement sous le nom \textbf{Machine Learning Grind}, dans lequel j’explore les fondements et les architectures de l’IA moderne à travers des lectures et implémentations régulières.

J’ai également complété cette expérience par deux stages dans des environnements appliqués : chez Innov+, j’ai conçu un système de prédiction d’émotions dans des vidéos à des fins de sécurité routière ; chez EDF, j’ai travaillé à la classification de défaillances de turbines à partir de rapports textuels, en appliquant des techniques de NLP et de fine-tuning de modèles pré-entraînés.

Aujourd’hui, je souhaite m’engager dans un travail de recherche approfondi, où je pourrai combiner mes compétences en IA, mes bases en mathématiques et mon intérêt fort pour la santé. La perspective d’une thèse me motive particulièrement, car elle représente à mes yeux une étape essentielle pour contribuer durablement à l’innovation en IA médicale, et construire à terme un projet entrepreneurial dans le domaine de la e-santé en Afrique.

Je me tiens à votre disposition pour échanger sur les sujets de recherche que vous portez, et vous remercie pour l’attention portée à ma candidature.
