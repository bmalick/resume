
\documentclass{src/classes/cover}

% Change the colours if you want to
\definecolor{Mulberry}{HTML}{72243D}
\definecolor{SlateGrey}{HTML}{2E2E2E}
\definecolor{LightGrey}{HTML}{666666}
\colorlet{heading}{Sepia}
\colorlet{accent}{Mulberry}
\colorlet{emphasis}{SlateGrey}
\colorlet{body}{LightGrey}

\definecolor{innov-plus-color}{HTML}{1b2129}


% [Post title]
\setpostname{Data Scientist - London}

\begin{document}

% [Header]
\coverHeader

Madame, Monsieur,

% [Introduction paragraph]
% Je serai diplômé de CentraleSupélec cette année, tout en étant titulaire d'un double diplôme avec le Master de recherche en Mathématiques Fondamentales Appliquées de l'Université de Lorraine. Ma spécialisation porte sur l'intelligence artificielle, et plus précisément sur ses applications en vision par ordinateur (\textit{computer vision}). Je souhaite ainsi intégrer votre équipe pour le poste de \postName, afin de contribuer activement à vos projets en intelligence artificielle et en traitement de données.
Diplômé cette année de CentraleSupélec avec un double diplôme en Master de recherche en Mathématiques Fondamentales Appliquées (Université de Lorraine), je souhaite rejoindre vos équipes R\&D en tant que Data Scientist. Mon parcours combine une formation rigoureuse en mathématiques appliquées et une expérience pratique significative en intelligence artificielle.
% et computer vision.
% Diplômé cette année de CentraleSupélec avec un Master de recherche en Mathématiques Fondamentales Appliquées, je candidate pour rejoindre vos équipes en tant que Data Scientist. Mon profil allie expertise mathématique avancée et expérience pratique en intelligence artificielle appliquée.

% [Post interest]
% Le poste de \postName\ m'intéresse particulièrement car il correspond à mes domaines de compétences et à mes centres d'intérêt, notamment en intelligence artificielle et en traitement de données. Je suis motivé par l'idée de contribuer à des projets concrets, avec un impact réel, tout en continuant à apprendre et à progresser. Travailler au sein de votre équipe représente pour moi une excellente opportunité de mettre mes connaissances en pratique dans un environnement stimulant et collaboratif.
Dassault Systèmes m'attire par son positionnement unique à l'intersection des mathématiques avancées et de l'innovation industrielle. Votre approche de la simulation numérique et du jumeau numérique, notamment dans l'aéronautique et l'automobile, résonne parfaitement avec mon goût pour les défis techniques complexes et les applications concrètes de la recherche.
% Decartes Underwriting m'inspire par son approche révolutionnaire de l'assurance paramétrique, où modélisation mathématique rigoureuse et innovation technologique se conjuguent pour répondre aux défis climatiques contemporains. Votre capacité à transformer des données complexes en solutions concrètes de gestion des risques correspond parfaitement à ma vision de l'impact positif des mathématiques appliquées.

% [Company specificity]

% [Why me ?]
% Grâce à ma formation d'ingénieur à CentraleSupélec, complétée par un master de recherche en mathématiques fondamentales appliquées, j'ai acquis des bases solides en mathématiques, en programmation et en intelligence artificielle. Mon parcours m'a permis de développer une approche rigoureuse, analytique et orientée vers la résolution de problèmes. Je suis autonome, curieux, et j'aime apprendre par la pratique ainsi que par la lecture d'ouvrages spécialisés en statistiques et en IA. Mon intérêt pour ce domaine se traduit par des projets personnels et académiques variés, notamment en vision par ordinateur (\textit{computer vision}), un sujet que je serais heureux de développer davantage lors d'un entretien. Je suis également à l'aise avec les outils de développement modernes (Python, Git, Docker, etc.), ce qui me permet de m'intégrer rapidement à un environnement technique.
% Mon expérience chez Michelin (réduction d'artefacts métalliques par deep learning sur images tomographiques) et chez EDF (classification des défaillances de turbines avec fine-tuning de CamemBERT) illustre ma capacité à appliquer des modèles de pointe à des problématiques industrielles réelles. Mon projet personnel "Machine Learning Grind" témoigne de ma passion pour la recherche : j'y implémente from scratch les architectures classiques (LeNet, AlexNet, VGG, GoogleNet, ResNet) et les entraîne sur ImageNet-100, démontrant une compréhension profonde des fondements mathématiques du deep learning.
J'ai mené plusieurs projets de lors de mes expériences professionnelles: chez Michelin, j'ai développé un benchmark de modèles de deep learning pour réduire les artefacts métalliques en imagerie tomographique ; chez EDF, j'ai adapté le modèle d'IA de texte CamemBERT pour classifier automatiquement les défaillances de turbines ; chez Innov+, j'ai conçu un modèle de prédiction d'émotions à partir de vidéos. Ces expériences illustrent ma capacité à transformer des concepts avancés en solutions opérationnelles.
% Mes expériences récentes démontrent cette approche : chez Michelin, j'ai développé des modèles de diffusion pour la réduction d'artefacts sur images tomographiques ; chez EDF, j'ai fine-tuné CamemBERT pour la classification automatique de défaillances industrielles. 
Parallèlement, mon projet \textit{Machine Learning Grind} reflète ma passion pour le domaine de l'intelligence artificielle: j'y implémente from scratch les architectures de référence et explore leurs fondements théoriques, cultivant une compréhension profonde des mécanismes d'apprentissage.
% Parallèlement, mon projet \textit{Machine Learning Grind} reflète ma passion pour la recherche fondamentale : j'y implémente from scratch les architectures de référence et explore leurs fondements théoriques, cultivant une compréhension profonde des mécanismes d'apprentissage.

% [Skills]
% Au fil de mon parcours, j'ai développé de solides compétences en mathématiques appliquées, notamment en probabilités et statistiques avancées, en optimisation, ainsi qu'en apprentissage automatique (\textit{machine learning}). J'ai consolidé ces compétences à travers mon master de recherche en mathématiques fondamentales appliquées et ma spécialisation de dernière année à CentraleSupélec en science des données et de l'information. Je maîtrise les principaux frameworks de \textit{machine learning} et \textit{deep learning} (Scikit-Learn, \textit{PyTorch}, \textit{TensorFlow}), ainsi que des bibliothèques et outils de traitement de données (\textit{pandas}, \textit{NumPy}). Je suis également à l'aise avec les environnements de développement modernes (\textit{Git}, \textit{Docker}, \textit{Conda}), ce qui me permet de m'intégrer rapidement dans des projets concrets et techniques. Mon projet personnel \textit{Machine Learning Grind}, que je maintiens sur GitHub, reflète mon investissement profond dans la compréhension et l'expérimentation des modèles d'IA.
% Ma double formation me permet d'allier rigueur théorique et pragmatisme technique, compétences essentielles pour contribuer à vos projets de R\&D en intelligence artificielle appliquée à l'ingénierie.
% Ma formation en processus stochastiques et probabilités avancées, couplée à cette expérience pratique, me positionne idéalement pour contribuer à vos modèles de risque climatique et développer des algorithmes robustes dans un contexte d'incertitude.

% [Conclusion]
% Je serais heureux de pouvoir échanger avec vous à propos de cette opportunité. Je reste bien entendu à votre disposition pour tout complément d'information ou entretien. Je vous remercie pour l'attention portée à ma candidature.
Je serais ravi d'échanger sur cette opportunité et de vous présenter mes travaux plus en détail.
% Je serais enthousiaste de discuter de votre vision et de mes contributions potentielles à vos projets.

Cordialement,

\vspace{1em} Baye Malick Gning

\end{document}
