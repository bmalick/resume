\documentclass{src/classes/cover}

% Change the colours if you want to
\definecolor{Mulberry}{HTML}{72243D}
\definecolor{SlateGrey}{HTML}{2E2E2E}
\definecolor{LightGrey}{HTML}{666666}
\colorlet{heading}{Sepia}
\colorlet{accent}{Mulberry}
\colorlet{emphasis}{SlateGrey}
\colorlet{body}{LightGrey}

\definecolor{innov-plus-color}{HTML}{1b2129}


\def\companySpecificFr{}

\def\medicalFocus{
Le secteur médical me passionne par sa dimension humaine et l'impact direct des innovations technologiques sur la qualité des soins. Contribuer à l'amélioration du diagnostic médical par l'IA représente pour moi une motivation profonde, alliant excellence technique et utilité sociétale.
}

% [Post title]
\setpostname{Data Scientist - London}

\begin{document}

% [Header]
\coverHeader
% \coverHeaderEn

Madame, Monsieur,

% [Introduction paragraph]
Je serai diplômé de CentraleSupélec cette année, tout en étant titulaire d'un double diplôme avec le Master de recherche en Mathématiques Fondamentales Appliquées de l'Université de Lorraine. Ma spécialisation porte sur l'intelligence artificielle, et plus précisément sur ses applications en vision par ordinateur (\textit{computer vision}) dans le domaine médical. Je souhaite ainsi intégrer votre équipe pour \postName, afin de contribuer à vos travaux de recherche à l'intersection de l'IA et de la santé.

% [Post interest]
Le sujet de thèse m'attire particulièrement par son potentiel impact sur dans domaine de la santé et sa capacité à transformer la recherche en solutions concrètes pour les professionnels de santé. L'opportunité de travailler sur des algorithmes d'analyse d'imagerie cardiaque représente pour moi un défi technique passionnant, tout en ayant un impact direct sur l'amélioration des soins aux patients.
 Je suis convaincu que l'intelligence artificielle jouera un rôle majeur dans les avancées médicales des prochaines années, et je souhaite contribuer à cette évolution en y apportant mes compétences et ma passion.
Je suis également très motivé par la perspective d'évoluer dans un environnement pluridisciplinaire, où la rigueur scientifique se conjugue à un fort impact sociétal. Participer à un projet aussi ambitieux représente pour moi une opportunité stimulante de mettre mes connaissances en IA au service de problématiques de santé concrètes.

% [Company specificity]

% [Why me ?]
Ma double formation en ingénierie à CentraleSupélec et en mathématiques fondamentales appliquées à l'Université de Lorraine m'a permis d'acquérir une solide compréhension théorique et pratique de l'intelligence artificielle. J'ai notamment développé des projets en vision par ordinateur (\textit{computer vision}) appliquée au domaine médical, ce qui m'a sensibilisé aux enjeux spécifiques de ce secteur.

Je suis autonome, curieux, et j'aime approfondir mes connaissances par la pratique ainsi que par la lecture d'ouvrages et d'articles sur l'IA appliquée à la santé. Je m'intéresse particulièrement aux revues comme \textit{Nature Biomedical Engineering}, \textit{npj Digital Medicine} ou encore \textit{The Lancet Digital Health}, qui m'ont permis de mieux comprendre l'impact des techniques d'IA dans le diagnostic médical et la médecine personnalisée. Je serais ravi de développer ces points lors d'un entretien. Je suis également à l'aise avec les outils de développement modernes (Python, Git, Docker, etc.), ce qui me permet de m'intégrer rapidement à un environnement technique et collaboratif.

% [Skills]
Au fil de mon parcours, j'ai développé de solides compétences en mathématiques appliquées, notamment en probabilités et statistiques avancées, en optimisation, ainsi qu'en apprentissage automatique (\textit{machine learning}). Ces compétences ont été consolidées à travers mon master de recherche en mathématiques fondamentales appliquées et ma spécialisation en science des données et de l'information à CentraleSupélec.

Je maîtrise les principaux frameworks de \textit{machine learning} et \textit{deep learning} (Scikit-Learn, \textit{PyTorch}, \textit{Tensorflow}), ainsi que des bibliothèques spécifiques au traitement d'images médicales (\textit{OpenCV}, \textit{Albumentations}). Ces outils m'ont permis de contribuer à plusieurs projets liés à la santé : localisation 3D de bactéries par holographie numérique (LEMTA), débruitage d'images de stents simulées, et réduction d'artefacts métalliques sur images tomographiques lors de mon stage chez Michelin.

Mon projet personnel \textit{Machine Learning Grind}, que je maintiens sur GitHub, reflète mon investissement profond dans la compréhension et l'expérimentation des modèles d'IA.

% [Conclusion]
Je serais ravi de pouvoir échanger avec vous sur votre projet de recherche et la manière dont je pourrais y contribuer. Je reste bien entendu à votre disposition pour un entretien ou toute information complémentaire.

Je vous remercie pour l'attention portée à ma candidature.

\vspace{1em} Baye Malick Gning
\end{document}
